\documentclass[a4paper,12pt]{article} 
% 使用ctex包支持中文
\usepackage{ctex}
\usepackage{amsmath}
% 开始文档
\usepackage{color}
\usepackage{graphicx}
\begin{document}

% 创建标题页的内容
\title {前川定理(The Maekawa-Justin Theorem)}
\author{潘世维}
\date{Saturday, August 15, 2020}
\maketitle
\section{简介} 
\includegraphics[scale=0.5]{D:/OneDrive/markdown/Maekawa/artist_jun_maekawa.jpg}

MAEKAWA Jun (前川淳),日本的软件工程师,数学家,折纸艺术家。

\section{平顶点折叠的例子}  

\includegraphics[scale=0.15]{D:/OneDrive/markdown/Maekawa/image-20200814130413151.png}
\includegraphics[scale=0.15]{D:/OneDrive/markdown/Maekawa/image-20200814130404199.png}
\includegraphics[scale=0.4]{D:/OneDrive/markdown/Maekawa/image-20200814130423490.png}


让$M(\text{Mountain})$和$V(\text{Valley})$分别代表平顶点折叠中山折和谷折的数量,则前川定理可以表示为

$M = V +2 \quad \text{or}\quad V = M + 2$

即$|M-V|=2$

$\text{Proof}\;1:$

![image-20200814124909047](前川定理/image-20200814124909047.png)

由于我们只关心顶点$x$和周围的折痕,所以可以以$x$为圆心做一个圆$(a)$,按折痕折叠后形成$(b)$

![image-20200814125047016](前川定理/image-20200814125047016.png)

从下往上看向顶点x,可以发现圆环形成了一个闭合回路$(c)$

![image-20200814125056916](前川定理/image-20200814125056916.png)

想象有一个蚂蚁从$p$点出发在这个闭合回路上爬行,遇到山折便逆时针旋转$180^{\circ}$,遇到谷折便顺时针旋转$180^{\circ}$,最后回到原点,方向和开始一样,由于沿着闭合回路走了一周,相当于旋转了$360^{\circ}$度,即

$M · 180^{\circ} +V ·(−180^{\circ}) = 360^{\circ}$

$M − V = 2$

因为纸有两面,如果从另一面看,原来的山折变成了谷折,原来的谷折变成了山折,所以有

$V − M = 2$

这样便证明了前川定理

$\text{Proof}\;2:$

这个证明是由$\text{Jan Siwanowicz}$在他还是个高中生的时候提出的

![image-20200814130533821](前川定理/image-20200814130533821.png)

将此前的闭合回路看作一个多边形,把山折看成内角等于$0$,谷折看成内角等于$360^{\circ}$

由多边形内角和定理

$\sum\limits_{i=1}^n\theta_i=(n-2)×180^{\circ}$

推得在这个多边形中,内角和为

$M · 0^{\circ} +V · 360^{\circ}$

所以$V · 360^{\circ} = (M +V-2)180^{\circ}$

$\therefore M = V +2 \quad \text{or}\quad V = M + 2$

![image-20200814191350681](前川定理/image-20200814191350681.png)

### 推广:

$M + V = 2(V+1) \quad \text{or} \quad2(V − 1)$

得到偶数定理:单顶点折叠中折痕总数必为偶数,角的总数也必为偶数
\begin{flushright}
    powered by \LaTeX\\
    made by Erikpsw\\ 
\end{flushright}

\end{document}  