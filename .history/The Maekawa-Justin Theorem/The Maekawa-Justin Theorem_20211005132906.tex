\documentclass[a4paper,12pt]{article} 
% 使用ctex包支持中文
\usepackage{ctex}
\usepackage{amsmath}
% 开始文档
\usepackage{graphicx}
\usepackage{subfigure}
\usepackage{caption}
\usepackage[table,xcdraw]{xcolor}
\captionsetup[figure]{name=Figure}
\begin{document}


% 创建标题页的内容
\title {前川定理(The Maekawa-Justin Theorem)}
\author{潘世维}
\date{Saturday, August 15, 2020}
\maketitle
\section{简介} 
\includegraphics[scale=0.6]{D:/OneDrive/markdown/Maekawa/artist_jun_maekawa.jpg}

MAEKAWA Jun (前川淳),日本的软件工程师,数学家,折纸艺术家。

\section{平顶点折叠的例子}  
\begin{flushright}
\begin{figure}[h]
    \centering 
    \begin{minipage}{0.4\linewidth}
    \centering    %子图居中
    \includegraphics[scale=0.2]{D:/OneDrive/markdown/Maekawa/image-20200814130413151.png}	 
    \caption{}
    \end{minipage}
    \hfill
    \begin{minipage}{0.4\linewidth}
    \centering    %子图居中
    \includegraphics[scale=0.2]{D:/OneDrive/markdown/Maekawa/image-20200814130404199.png}
    \caption{}
    \end{minipage}
    \includegraphics[scale=0.3]{D:/OneDrive/markdown/Maekawa/image-20200814130423490.png}
    \caption{}
\end{figure}
\end{flushright}

\begin{table}[]
    \centering
    \begin{tabular}{|l|c|l|l|c|l|c|}
    \hline
     & \multicolumn{3}{c|}{$M$(Mountain)} & \multicolumn{2}{c|}{$V$(valley)} & \multicolumn{1}{l|}{$M-V$} \\ \hline
    Figure 1 & \multicolumn{3}{c|}{1} & \multicolumn{2}{c|}{3} & {\color[HTML]{9A0000} 2} \\ \hline
    Figure 2 & \multicolumn{3}{c|}{4} & \multicolumn{2}{c|}{2} & {\color[HTML]{9A0000} 2} \\ \hline
    Figure 3 & \multicolumn{3}{c|}{5} & \multicolumn{2}{c|}{3} & {\color[HTML]{9A0000} 2} \\ \hline
    \end{tabular}
\end{table}
      




\section{前川定理的证明\cite{demaine2007geometric}}
让$M(\text{Mountain})$和$V(\text{Valley})$分别代表平顶点折叠中山折和谷折的数量,则前川定理可以表示为
\begin{equation}\label{2}
    M = V +2 \quad \text{or}\quad V = M + 2
\end{equation}
    


即$|M-V|=2$

\subsection{$\text{Proof}\;1:$}
\begin{center}
\includegraphics[scale=0.25]{D:/OneDrive/markdown/Maekawa/image-20200814124909047.png}
\end{center}

由于我们只关心顶点$x$和周围的折痕,所以可以以$x$为圆心做一个圆$(a)$,按折痕折叠后形成$(b)$\\

\begin{center}
\includegraphics[scale=0.3]{D:/OneDrive/markdown/Maekawa/image-20200814125047016.png}\\
\end{center}

从下往上看向顶点$x$,可以发现圆环形成了一个闭合回路$(c)$

\begin{center}
    \includegraphics[scale=0.3]{D:/OneDrive/markdown/Maekawa/image-20200814125056916.png}
\end{center}


想象有一个蚂蚁从$p$点出发在这个闭合回路上爬行,遇到山折便逆时针旋转$180^{\circ}$,遇到谷折便顺时针旋转$180^{\circ}$,最后回到原点,方向和开始一样,由于沿着闭合回路走了一周,相当于旋转了$360^{\circ}$度,即
\begin{equation}\label{1}
    M \times 180^{\circ} +V \times (-180^{\circ}) = 360^{\circ}
\end{equation}


$M - V = 2$

因为纸有两面,如果从另一面看,原来的山折变成了谷折,原来的谷折变成了山折,所以有

$V - M = 2$

这样便证明了前川定理

\subsection{$\text{Proof}\;2:$}

这个证明是由$\text{Jan Siwanowicz}$在他还是个高中生的时候提出的
\begin{center}
\includegraphics[scale=0.3]{D:/OneDrive/markdown/Maekawa/image-20200814130533821.png}\\
\end{center}

将此前的闭合回路看作一个多边形,把山折看成内角等于$0^{\circ}$,谷折看成内角等于$360^{\circ}$

由多边形内角和定理

$\sum\limits_{i=1}^n\theta_i=(n-2)×180^{\circ}$

推得在这个多边形中,内角和为

$M \times 0^{\circ} +V \times 360^{\circ}$

所以$V\times 360^{\circ} = (M +V-2)180^{\circ}$

$ M = V +2 \quad \text{or}\quad V = M + 2$
\begin{center}
\includegraphics[scale=0.3]{D:/OneDrive/markdown/Maekawa/image-20200814191350681.png}\\
\end{center}

\section{推广}
\begin{equation}\label{3}
    M + V = 2(V+1) \quad \text{or} \quad2(V- 1)
    \end{equation}

得到偶数定理:单顶点折叠中折痕总数必为偶数,角的总数也必为偶数
\begin{flushright}
    powered by \LaTeX\\
    made by Erikpsw\\ 
\end{flushright}

\bibliographystyle{plain}
\bibliography{ref}
\end{document}  