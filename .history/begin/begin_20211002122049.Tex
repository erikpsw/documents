\documentclass[UTF8]{article}
\usepackage{ctex}
\usepackage{Mhchem}
\begin{document}       
 
\section{开始}   
测试       
\LaTeX \,\\ 
 $a^2+b^2=c^2$\\
 $\begin{aligned}
    \left|P-P_{n}\right|&=\left|a \sum_{k=n}^{\infty}\left(a e^{\frac{\pi}{4} i}\right)^{k}\right|=\left|a\left(a e^{\frac{\pi}{4} i}\right)^{n} \sum_{k=0}^{\infty}\left(a e^{\frac{\pi}{4} i}\right)^{k}\right|\\&=\left|(2-\sqrt{2})((\sqrt{2}-1)(1+i))^{n}\right| \cdot|P|\\&=(2-\sqrt{2})((\sqrt{2}-1) \sqrt{2})^{n} \frac{\sqrt{6}}{3}\\&=\frac{\sqrt{6}}{3}(2-\sqrt{2})^{n+1}\end{aligned}$

\subsection{例子}  

量子效应\cite{Can}
\\
\ce{2H2 + O2 = 2H2O}
$$\begin{array}{|c|c|c|c|c|c|}
    \hline
    \text{character}&\text{a}&\text{b}&\text{c}&\text{d}&\text{e}\\
    \hline
    f(c)&45&13&12&16&9\\
    \hline
    \text{Huffman Code}&0&101&100&111&1101\\
    \hline
    \end{array}$$
$$P\left(\frac{X_{1}+X_{2}+\cdots+X_{n}-n \mu}{\sigma \sqrt{n}} \leq x\right)\to \frac{1}{\sqrt{2 \pi}} \int_{-\infty}^{x} e^{-t^{2} / 2} d t$$



$$
\cos\theta=\frac{\vec{\rm{n_1}}\cdot\vec{\rm{n_2}}  }{|\vec{\rm{n_1}}|\cdot|\vec{\rm{n_2}}|}=\frac{x_{1} x_{2}+y_{1} y_{2}+z_{1} z_{2}}{\sqrt{x_{1}^{2}+y_{1}^{2}+z_{1}^{2}} \sqrt{x_{2}^{2}+y_{2}^{2}+z_{2}^{2}}}
$$
\section{人工转变}

$$
\begin{cases}\rm
    _{2}^{4}He+ _{7}^{14}N \rightarrow _{8}^{17}O + _{1}^{1}H(Rutherford)\\\rm
    _{2}^{4}He+ _{4}^{8}Be \rightarrow _{6}^{11}C + _{0}^{1}n(Chadwick)\\\rm
    _{2}^{4}He+ _{13}^{27}Al \rightarrow _{15}^{30}P + _{0}^{1}n,_{15}^{30}P \rightarrow _{14}^{30}Si + _{1}^{0}e(Curie)
    \end{cases}
$$

$$
{C_xH_yO_z +(x +\frac y4-\frac z2)O_2\rightarrow xCO_2+\frac y2H_2O}
$$


\section{Part 2}
关于$e^{n+1}$中超球面特征的一些结果 \cite{Wen}

$$
\text{Proof:}\quad{e^{\sqrt {{x_1} \cdot {x_2}} }} < \frac{{{e^{{x_2}}} - {e^{{x_1}}}}}{{{x_2} - {x_1}}}
\quad({x_2} > {x_1})$$
$$
{e^{\sqrt {{x_1} \cdot {x_2}} }} < \frac{{{e^{{x_2}}} - {e^{{x_1}}}}}{{{x_2} - {x_1}}}
\quad({x_2} > {x_1})
\Leftrightarrow 
\frac{{{e^{{x_2} - \sqrt {{x_1} \cdot {x_2}} }} - {e^{{x_1} - \sqrt {{x_1} \cdot {x_2}} }}}}{{{x_2} - {x_1}}} > 1
$$
$$\begin{aligned}
\frac{{{e^{{x_2} - \sqrt {{x_1} \cdot {x_2}} }} - {e^{{x_1} - \sqrt {{x_1} \cdot {x_2}} }}}}{{{x_2} - {x_1}}}&=\frac{{{e^{{x_2} - \sqrt {{x_1} \cdot {x_2}} }} - {e^{{x_1} - \sqrt {{x_1} \cdot {x_2}} }}}}{{({x_2} - \sqrt {{x_1} \cdot {x_2}} ) - ({x_1} - \sqrt {{x_1} \cdot {x_2}} )}}\\
&=\frac{{{e^{{x_2} - \sqrt {{x_1} \cdot {x_2}} }} - {e^{{x_1} - \sqrt {{x_1} \cdot {x_2}} }}}}{{\ln ({e^{({x_2} - \sqrt {{x_1} \cdot {x_2}} )}}) - \ln ({e^{({x_1} - \sqrt {{x_1} \cdot {x_2}} )}})}}\\
&>\sqrt {{e^{{x_2} - \sqrt {{x_1} \cdot {x_2}}  + {x_1} - \sqrt {{x_1} \cdot {x_2}} }}}\\
&=\sqrt {{e^{{{(\sqrt {{x_2}}  - \sqrt {{x_1}} )}^2}}}}>\sqrt {e^0}=1
\end{aligned}
$$

$\text{Rt}\triangle ABC$中,$\angle A=\frac\pi2$, $AM$为中线,$AD$为角平分线

以$\vec{AB}$为$x$轴,$\vec{AC}$为$y$轴建系,不妨设$AB=a$,$AC=b$

由中点公式
$$
\left\{ \begin{gathered}
  {x_M} = \frac{{{x_B} + {x_C}}}{2} \hfill \\
  {y_M} = \frac{{{y_B} + {y_C}}}{2} \hfill \\\end{gathered}  \right.\Rightarrow M\left( {\frac{a}{2},\frac{b}{2}} \right)
$$
$ BC$的截距式方程为$\frac{x}{a} + \frac{y}{b} = 1$,与$y=x$联立得
$$
D\left( {\frac{{ab}}{{a + b}},\frac{{ab}}{{a + b}}} \right)
$$

$$
\begin{aligned}
 A{M^2} &= \frac{{{a^2}}}{4} + \frac{{{b^2}}}{4}\\
	&={\sqrt {\frac{{{{\left( {\frac{a}{{\sqrt 2 }}} \right)}^2} + {{\left( {\frac{b}{{\sqrt 2 }}} \right)}^2}}}{2}} ^2}
\\AD^2&=2{\left( {\frac{{ab}}{{a + b}}} \right)^2}\\
&= {\left( {\frac{2}{{\frac{{\sqrt 2 }}{a} + \frac{{\sqrt 2 }}{b}}}} \right)^2}
\end{aligned}
$$

已知
$$
\begin{array}{c} 
  H_{n}=\frac{n}{\sum \limits_{i=1}^{n}\frac{1}{x_{i}}}= \frac{n}{\frac{1}{x_{1}}+ \frac{1}{x_{2}}+ \cdots + \frac{1}{x_{n}}} \\ Q_{n}=\sqrt{\sum \limits_{i=1}^{n}x_{i}^{2}}= \sqrt{\frac{x_{1}^{2}+ x_{2}^{2}+ \cdots + x_{n}^{2}}{n}} \\ H_{n}\leq Q_{n} 
\end{array}
$$
由$H_2\le Q_2$,
得
$$
AM\ge AD
$$
等号当且仅当$AB=AC$时成立

另解,由中线定理
$$
{\left( {2AM} \right)^2} + B{C^2} = 2(A{B^2} + A{C^2}$$
$$AM = \frac{{2({a^2} + {b^2}) - ({a^2} + {b^2})}}{4} = \frac{{{a^2}}}{4} + \frac{{{b^2}}}{4}\\
$$
由$S_{\triangle ABC}=S_{\triangle ABD}+S_{\triangle ACD}$
$$
\frac{1}{2}AB \cdot AC = \frac{1}{2}AB \cdot AD \cdot \sin \frac{\pi }{4} + \frac{1}{2}AC \cdot AD \cdot \sin \frac{\pi }{4}$$
$$
AD = \frac{{\sqrt 2 AB \cdot AC}}{{AB + AC}} = \frac{{\sqrt 2 ab}}{{a + b}}
$$

以$F_1$为极点,垂直于左准线方向为极轴建系
$$
L = \rho _\theta + \rho _{\pi  - \theta}  = \left| \frac{2ep}{ {1 - {e^2}\cos ^2}\theta } \right|
$$

$$
\begin{aligned}
{S_{PMQN}}& = \frac{1}{2}|PQ| \cdot |MN|\\
& = \frac{1}{2} \cdot \frac{{2ep}}{{1 - {e^2}{{\cos }^2}\theta }} \cdot \frac{{2ep}}{{1 - {e^2}{{\cos }^2}(\pi-\theta)}}\\
&=\frac{{8{e^2}{p^2}}}{{4 - 4{e^2} + {e^4}{{\sin }^2}2\theta }}
\end{aligned}$$
$$
 e=\frac{c}{a},p=\frac{{{a^2}}}{c} - c = \frac{{{b^2}}}{c}
 $$
 $${S_{{\text{min}}}}=\frac{{8{e^2}{p^2}}}{{4 - 4{e^2} + {e^4}}}=\frac{{8{a^2}{b^4}}}{{{{({a^2} + {b^2})}^2}}}
 $$
 $${S_{{\text{max}}}} = \frac{{2{e^2}{p^2}}}{{1 - {e^2}}} = 2{b^2}
$$



另解

设$PQ$方程为$x=my-c$,联立
$$
\begin{cases}
\frac{{{x^2}}}{{{a^2}}} + \frac{{{y^2}}}{{{b^2}}} = 1\\
x=my-c
\end{cases} \Rightarrow ({a^2} + {b^2}{m^2}){y^2} - 2mc{b^2}y - {b^2}{c^2} = 0
$$

$$
|PQ| = \sqrt {1 + {m^2}}  \cdot |{y_1} - {y_2}| = \frac{{2a{b^2}({m^2} + 1)}}{{{a^2} + {b^2}{m^2}}}
$$

用$ - \frac{1}{m}$替换$m$
$$
|MN| = \frac{{2a{b^2}\left[{{\left( - \frac{1}{m}\right)}^2} + 1\right]}}{{{a^2} + {b^2}{{( - \frac{1}{m})}^2}}} = \frac{{2a{b^2}\left({m^2} + 1\right)}}{{{a^2}{m^2} + {b^2}}}\\
$$
令$t=m^2$
$$
\begin{aligned}
 {S_{PMQN}}& = \frac{1}{2}|PQ| \cdot |MN|\\
& = \frac{{2{a^2}{b^4}{{({t} + 1)}^2}}}{{({a^2} + {b^2}{t})({a^2}{t} + {b^2})}}\\
&=\frac{{2{a^2}{b^4}({t^2} + 2t + 1)}}{{{a^2}{b^2}{t^2} + ({a^4} + {b^4})t + {a^2}{b^2}}}
\end{aligned}\\
$$
若$t=0\Rightarrow S= 2{b^2}$

若$t \ne   0$,分离常数
$$
\begin{aligned}
 {S_{PMQN}}& = 2{a^2}{b^4} \cdot \left[ {\frac{{\left( {2 - \frac{{{a^4} + {b^4}}}{{{a^2}{b^2}}}} \right)t}}{{{a^2}{b^2}{t^2} + ({a^4} + {b^4})t + {a^2}{b^2}}} + \frac{1}{{{a^2}{b^2}}}} \right]
 \\&=2{b^2} - \frac{{2({a^2} - {b^2}){b^2}}}{{{a^2}{b^2}\left( {t + \frac{1}{t}} \right) + {a^4} + {b^4}}}
 \\&\ge 2{b^2} - \frac{{2({a^2} - {b^2}){b^2}}}{{{{({a^2} + {b^2})}^2}}}
 \\&=\frac{{8{a^2}{b^4}}}{{{{({a^2} + {b^2})}^2}}}
\end{aligned}
$$
当且仅当$t=m^2=1,k=\pm 1$ 时取等
$$
\lim_{t \to 0/+\infty} \frac{{2({a^2} - {b^2}){b^2}}}{{{a^2}{b^2}\left( {t + \frac{1}{t}} \right) + {a^4} + {b^4}}}=0
$$

$$
  {S_{{\text{min}}}}=\frac{{8{a^2}{b^4}}}{{{{({a^2} + {b^2})}^2}}}
  $$
$$
  {S_{{\text{max}}}}  = 2{b^2}
$$

注
$$
\frac{{2{a^2}{b^4}{{({t} + 1)}^2}}}{{({a^2} + {b^2}{t})({a^2}{t} + {b^2})}}\ge\frac{{2{a^2}{b^4}{{(t + 1)}^2}}}{{\frac{1}{4} \cdot {{[({a^2} + {b^2})t + ({a^2} + {b^2})]}^2}}}=\frac{{8{a^2}{b^4}}}{{{{({a^2} + {b^2})}^2}}}
$$

\bibliographystyle{plain}
\bibliography{ref}

\end{document}  



