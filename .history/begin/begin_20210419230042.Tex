\documentclass[UTF8]{article}
\usepackage{ctex}
\usepackage[version=4]{mhchem}
\usepackage{amsmath}
\begin{document}       
 
\section{开始}   
使用       
\LaTeX \,\\ 
 $a^2+b^2=c^2$\\
 $\begin{aligned}
    \left|P-P_{n}\right|&=\left|a \sum_{k=n}^{\infty}\left(a e^{\frac{\pi}{4} i}\right)^{k}\right|=\left|a\left(a e^{\frac{\pi}{4} i}\right)^{n} \sum_{k=0}^{\infty}\left(a e^{\frac{\pi}{4} i}\right)^{k}\right|\\&=\left|(2-\sqrt{2})((\sqrt{2}-1)(1+i))^{n}\right| \cdot|P|\\&=(2-\sqrt{2})((\sqrt{2}-1) \sqrt{2})^{n} \frac{\sqrt{6}}{3}\\&=\frac{\sqrt{6}}{3}(2-\sqrt{2})^{n+1}\end{aligned}$

\subsection{例子}  

量子效应\cite{Can}
\\
\ce{2H2 + O2 -> 2H2O}
$$\begin{array}{|c|c|c|c|c|c|}
    \hline
    \text{character}&\text{a}&\text{b}&\text{c}&\text{d}&\text{e}\\
    \hline
    f(c)&45&13&12&16&9\\
    \hline
    \text{Huffman Code}&0&101&100&111&1101\\
    \hline
    \end{array}$$
$$P\left(\frac{X_{1}+X_{2}+\cdots+X_{n}-n \mu}{\sigma \sqrt{n}} \leq x\right)\to \frac{1}{\sqrt{2 \pi}} \int_{-\infty}^{x} e^{-t^{2} / 2} d t$$



$$
\cos\theta=\frac{\vec{\rm{n_1}}\cdot\vec{\rm{n_2}}  }{|\vec{\rm{n_1}}|\cdot|\vec{\rm{n_2}}|}=\frac{x_{1} x_{2}+y_{1} y_{2}+z_{1} z_{2}}{\sqrt{x_{1}^{2}+y_{1}^{2}+z_{1}^{2}} \sqrt{x_{2}^{2}+y_{2}^{2}+z_{2}^{2}}}
$$
\section{人工转变}

$$
\begin{cases}\rm
    _{2}^{4}He+ _{7}^{14}N \rightarrow _{8}^{17}O + _{1}^{1}H(Rutherford)\\\rm
    _{2}^{4}He+ _{4}^{8}Be \rightarrow _{6}^{11}C + _{0}^{1}n(Chadwick)\\\rm
    _{2}^{4}He+ _{13}^{27}Al \rightarrow _{15}^{30}P + _{0}^{1}n,_{15}^{30}P \rightarrow _{14}^{30}Si + _{1}^{0}e(Curie)
    \end{cases}
$$


\ce{C2H5OH\xrightleftharpoons[还原\ce{H2}]{氧化\ce{KMnO4(H+)/K2Cr2O7(H+)/O2}}CH3CHO\xrightarrow{氧化}CH3COOH}

\section{Part 2}
关于$E^{n+1}$中超球面特征的一些结果

\bibliographystyle{plain}
\bibliography{ref}

\end{document}  



