\documentclass[UTF8]{article}
\usepackage{ctex}

\begin{document}  

\title{Hello World}
我的的寒假化学论文

德国化学家舒尔茨曾赞誉道:“自然界除水之外, 氮是生长、发展和创造最有力的推动者。”虽然氮气在空气中占据最多的比例,但是我们却无法直接通过呼吸来摄入氮元素,只能通过食物来满足人体对必需常量元素氮的需求。

生态系统中的氮元素通常以氨或氨盐的形式被固定,经过硝化作用形成亚硝酸盐或硝酸盐,然后被绿色植物吸收并转化成为氨基酸,从而合成植物蛋白,食草动物可利用植物蛋白质合成动物蛋白质。人类通过对植物蛋白或动物蛋白的摄入,从而满足自身对氮元素的需求。最后动物的排泄物和动植物残体经细菌的分解作用形成氨、$CO_2$和水,排放到土壤中的氨又经细菌的硝化作用形成硝酸盐,被植物再次吸收、利用合成蛋白质,从而构成完整的氮循环。 但是,生态系统中氮大多以氮气的形式存在,为了让空气中的氮变得可以食用,自然界中只能依靠固氮细菌来完成氮气到植物所需要的亚硝酸盐的转变。同时,少量的动物的排泄物也被应用在碳循环中,如给庄稼施农家肥,但远远达不到人类需求。

直到合成氨的出现,打破了这一僵局,加快氮循环的同时也促进了人类的发展。

1774年,英国化学家普利斯特利首次通过加热氯化铵与熟石灰制备了氨气,而且在那个时候他已经知道了氨气显碱性或者至少溶解在水里呈碱性,因此他把氨气叫做碱空气。十年以后也就是1784年,还是法国科学家,贝托莱证明了氨是由氢与氮组成的。

十九世纪中期,也就是鸦片战争前后,随着农业科学的发展,人们已经认识到氮源对植物生长的重大意义,有意识地使用氨作为人工氮源提升农产品产量。这时候的氨来自于煤化工,因为煤里含有$\rm 1\%\sim2\%$的氮,在炼焦过程中,这些氮会转化为氨气,存在于煤气中,将这些煤气通入水中或者用硫酸吸收就得到了硫酸铵。这可以说这是人类最早制备铵肥的方法。

到了1898年,德国化学家首先采用化学方法完成了合成氨,其具体的合成方法是把电石$CaC_2$ 与氮气在1000摄氏度的高温下加热合成氰氨化钙  ,即石灰氮,反应方程式为:
$$
\rm CaC_2 + N_2\overset{\triangle }{\rightarrow}  CaCN_2 + C
$$

再用过热水蒸汽进行水解,分解为碳酸钙与氨气。
$$
\rm CaCN_2+3H_2O \rightarrow2NH_3+CaCO_3
$$

这种方法在20世纪头十年内非常盛行,年产量一度达到50万吨左右。但是这种方法缺点还是非常大的,电石不便宜,而且反应温度高,反应结束后还有碳酸钙作为废料产生。

就在全人类都开始迷茫的时候,热力学终于登场了,热力学大发展在19世纪中叶时期,其中热力学第二定律主要用于预测热力学体系发生变化的方向与可以到达的程度。在一系列努力下,人们开始尝试采用热力学预测化学反应发生的可能性与程度。为这方面做出突出贡献的是吉布斯与亥姆霍兹,他们相继提出了自由能的概念。
$$
\Delta G=\Delta H-T\Delta S
$$

终于时间来到了1908年,合成氨的突破终于来临。德国化学家哈伯,通过一系列的计算预测了不同温度,不同压力下合成氨的转化率与平衡浓度,随后又通过大量实验进行验证。通过以上工作哈伯认识到,过去之所以采用氮气与氢气直接合成无法取得良好的效果,主要归咎于以下几个原因。首先由于热力学的限制,这个反应单程转化率非常低,为了提高整体转化率,必须让反应气体在高压高温下进行循环,同时在循环的过程中还要想方法将氨气进行分离。其次,反应活化能非常高,反应速度非常慢,因此需要配合有效的催化剂,才能经济地进行合成氨反应。
$$
\rm N_2 +3H_2\rightleftharpoons  2NH_3
$$

反应所需的氢气主要来源于固体燃料、重质烃、轻质烃或气体烃加热至高温并与水蒸气反应生产含氢和一氧化碳为主的水煤气。第一步先把原料中的硫化物清除,是因为硫化物会毒害哈伯法所使用的催化剂。催化加氢可以把有机硫化物(如硫醇)变成硫化氢
$$
\rm H_2+ RSH → RH + H_2S(g)
$$
产生的硫化氢会被氧化锌吸收,变成水和硫化锌:
$$
\rm H_2S+ZnO\rightarrow ZnS+H_2O
$$
在镍的催化下与水反应,经脱硫的碳氢化合物(如甲烷)转变成氢气和一氧化碳的混合物:
$$
\rm CH_4+ H_2O → CO + 3 H_2
$$

$$
\rm C_nH_{2n+2}+nH_2O\rightarrow nCO+(2n+1)H_2
$$

一氧化碳进一步与水蒸气变换为氢气和二氧化碳:
$$
\rm CO + H_2O \rightleftharpoons CO_2+ H_2
$$

制备氢的最后步骤是以使用催化剂的甲烷化移除在氢气中残留的少量一氧化碳及二氧化碳:
$$
\rm CO+3H_2\rightarrow CH_4+H_2O
$$

$$
\rm CO_2+4H_2\rightarrow CH_4+2H_2O
$$

通过液化并分馏空气除去氧气得到氮气。得到的合成气还需经过纯化将残余的硫和碳的化合物脱除以防止催化剂中毒,即原料气的净化。之后合成气经过压缩达到合成氨需要的压力,最后送进反应塔进行反应,由于合成氨的转化率较低,原料气可以回收再利用。

在1913年9月9日哈伯实现了氨生产的工业化,起初氨的日产量只有$3\sim5t$。随着工艺条件的改善,1917年的氨年产量已超过$60000t$。凭借合成氨对人类做出的巨大贡献,哈伯获得了1918年度诺贝尔化学奖。

哈伯的一生非常具有争议性,一方面他是合成氨的创始人,另一方面他是第一个在战争中提出使用化学武器的人,这次行动导致了约2万人伤亡,主流上还是对他呈批判态度的。但是回顾历史,枪弹与核弹杀死的人远远多于化学武器,被燃烧弹烧死者死状更是悲惨百倍,但是这些武器的发明人大都得以善终。哈伯本人是犹太人,因此希特勒上台后也受尽迫害,最终死于逃亡的路上。

\end{document}  